\documentclass{article}%
\usepackage[T1]{fontenc}%
\usepackage[utf8]{inputenc}%
\usepackage{lmodern}%
\usepackage{textcomp}%
\usepackage{lastpage}%
\usepackage{geometry}%
\geometry{tmargin=4cm,lmargin=3cm,rmargin=3cm}%
%
%
%
\begin{document}%
\normalsize%
\section{Arztbrief}%
\label{sec:Arztbrief}%
Der Patient wurde uns aus der Notaufnahme zugewiesen. Bei Aufnahme präsentierte er sich wach und ansprechbar, jedoch mit Dyspnoe und einer Sauerstoffsättigung zwischen 65{-}75\% unter 8l Sauerstoff über eine Maske. Der Blutdruck betrug 31mmHg und die Herzfrequenz lag bei 31 Schlägen pro Minute. Nach der Erstversorgung in der Notaufnahme, die unter anderem eine Inhalation mit Salbutamol und Atrovent sowie eine arterielle und venöse Katheterlegung im Schockraum umfasste, wurde der Patient auf die Normalstation übernommen. 
\newline%

\newline%
Auf Station benötigte der Patient zunächst eine nicht{-}invasive Beatmung (NIV) mit einem PEEP von 5, um eine ausreichende Sauerstoffsättigung von 93\% zu gewährleisten. Im Verlauf zeigte sich eine Besserung der Atmungssituation, sodass der Patient morgens von der NIV auf eine Sauerstoffbrille umgestellt werden konnte. 
\newline%

\newline%
Aufgrund eines stark obstruktiven Atemmusters erhielt der Patient wiederholte Inhalationen mit Salbutamol und Atrovent über einen Spacer mit Maske sowie eine EzPAP{-}Therapie für 10 Minuten. Zur Ermöglichung eines Wechsels zwischen NIV und High{-}Flow{-}Therapie (HFOT) wurde das Beatmungssystem auf aktive Befeuchtung umgestellt. 
\newline%

\newline%
Im weiteren Verlauf wurde der Patient erfolgreich für eine HFOT behandelt und anschließend wieder an die NIV angeschlossen. 
\newline%

%
\end{document}