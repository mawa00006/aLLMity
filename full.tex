\documentclass{article}%
\usepackage[T1]{fontenc}%
\usepackage[utf8]{inputenc}%
\usepackage{lmodern}%
\usepackage{textcomp}%
\usepackage{lastpage}%
\usepackage{geometry}%
\geometry{tmargin=4cm,lmargin=3cm,rmargin=3cm}%
%
%
%
\begin{document}%
\normalsize%
\section{Arztbrief}%
\label{sec:Arztbrief}%
Der Patient, bekannt mit Adenokarzinom des Kolons, wurde uns aus der Notaufnahme mit dem Verdacht auf eine Verschlechterung seines Allgemeinzustandes zugewiesen. Er war zu Hause auffällig geworden durch massive Diarrhoe sowie einen Sturz mit der Notwendigkeit der Gabe von Sauerstoff durch den Rettungsdienst.
\newline%

\newline%
Bei Aufnahme zeigte sich ein Patient in deutlich reduziertem AZ, jedoch ohne abdominelle Symptomatik wie Schmerzen oder Abwehrspannung. Laborchemisch zeigte sich eine Entzündungskonstellation. Es erfolgte eine kalkulierte Antibiose sowie symptomatische Therapie. Der Patient entwickelte während des stationären Aufenthaltes eine Bradykardie mit Hypotonie und Hypoxämie, so dass er am 2024{-}01{-}28 auf die Intensivstation verlegt werden musste. Am 2024{-}01{-}23 konnte der Patient in deutlich gebessertem AZ in die Häuslichkeit entlassen werden. Es erfolgte die Planung einer ambulanten Weiterversorgung durch einen Pflegedienst. 
\newline%

%
\end{document}